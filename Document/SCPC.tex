% Options for packages loaded elsewhere
\PassOptionsToPackage{unicode}{hyperref}
\PassOptionsToPackage{hyphens}{url}
\PassOptionsToPackage{dvipsnames,svgnames,x11names}{xcolor}
%
\documentclass[
]{article}

\usepackage{amsmath,amssymb}
\usepackage{iftex}
\ifPDFTeX
  \usepackage[T1]{fontenc}
  \usepackage[utf8]{inputenc}
  \usepackage{textcomp} % provide euro and other symbols
\else % if luatex or xetex
  \usepackage{unicode-math}
  \defaultfontfeatures{Scale=MatchLowercase}
  \defaultfontfeatures[\rmfamily]{Ligatures=TeX,Scale=1}
\fi
\usepackage{lmodern}
\ifPDFTeX\else  
    % xetex/luatex font selection
\fi
% Use upquote if available, for straight quotes in verbatim environments
\IfFileExists{upquote.sty}{\usepackage{upquote}}{}
\IfFileExists{microtype.sty}{% use microtype if available
  \usepackage[]{microtype}
  \UseMicrotypeSet[protrusion]{basicmath} % disable protrusion for tt fonts
}{}
\makeatletter
\@ifundefined{KOMAClassName}{% if non-KOMA class
  \IfFileExists{parskip.sty}{%
    \usepackage{parskip}
  }{% else
    \setlength{\parindent}{0pt}
    \setlength{\parskip}{6pt plus 2pt minus 1pt}}
}{% if KOMA class
  \KOMAoptions{parskip=half}}
\makeatother
\usepackage{xcolor}
\setlength{\emergencystretch}{3em} % prevent overfull lines
\setcounter{secnumdepth}{5}
% Make \paragraph and \subparagraph free-standing
\ifx\paragraph\undefined\else
  \let\oldparagraph\paragraph
  \renewcommand{\paragraph}[1]{\oldparagraph{#1}\mbox{}}
\fi
\ifx\subparagraph\undefined\else
  \let\oldsubparagraph\subparagraph
  \renewcommand{\subparagraph}[1]{\oldsubparagraph{#1}\mbox{}}
\fi


\providecommand{\tightlist}{%
  \setlength{\itemsep}{0pt}\setlength{\parskip}{0pt}}\usepackage{longtable,booktabs,array}
\usepackage{calc} % for calculating minipage widths
% Correct order of tables after \paragraph or \subparagraph
\usepackage{etoolbox}
\makeatletter
\patchcmd\longtable{\par}{\if@noskipsec\mbox{}\fi\par}{}{}
\makeatother
% Allow footnotes in longtable head/foot
\IfFileExists{footnotehyper.sty}{\usepackage{footnotehyper}}{\usepackage{footnote}}
\makesavenoteenv{longtable}
\usepackage{graphicx}
\makeatletter
\def\maxwidth{\ifdim\Gin@nat@width>\linewidth\linewidth\else\Gin@nat@width\fi}
\def\maxheight{\ifdim\Gin@nat@height>\textheight\textheight\else\Gin@nat@height\fi}
\makeatother
% Scale images if necessary, so that they will not overflow the page
% margins by default, and it is still possible to overwrite the defaults
% using explicit options in \includegraphics[width, height, ...]{}
\setkeys{Gin}{width=\maxwidth,height=\maxheight,keepaspectratio}
% Set default figure placement to htbp
\makeatletter
\def\fps@figure{htbp}
\makeatother
\newlength{\cslhangindent}
\setlength{\cslhangindent}{1.5em}
\newlength{\csllabelwidth}
\setlength{\csllabelwidth}{3em}
\newlength{\cslentryspacingunit} % times entry-spacing
\setlength{\cslentryspacingunit}{\parskip}
\newenvironment{CSLReferences}[2] % #1 hanging-ident, #2 entry spacing
 {% don't indent paragraphs
  \setlength{\parindent}{0pt}
  % turn on hanging indent if param 1 is 1
  \ifodd #1
  \let\oldpar\par
  \def\par{\hangindent=\cslhangindent\oldpar}
  \fi
  % set entry spacing
  \setlength{\parskip}{#2\cslentryspacingunit}
 }%
 {}
\usepackage{calc}
\newcommand{\CSLBlock}[1]{#1\hfill\break}
\newcommand{\CSLLeftMargin}[1]{\parbox[t]{\csllabelwidth}{#1}}
\newcommand{\CSLRightInline}[1]{\parbox[t]{\linewidth - \csllabelwidth}{#1}\break}
\newcommand{\CSLIndent}[1]{\hspace{\cslhangindent}#1}

\usepackage[figon, printfigures]{figcaps}
\usepackage{booktabs}
\usepackage{longtable}
\usepackage{array}
\usepackage{multirow}
\usepackage{wrapfig}
\usepackage{float}
\usepackage{colortbl}
\usepackage{pdflscape}
\usepackage{tabu}
\usepackage{threeparttable}
\usepackage{threeparttablex}
\usepackage[normalem]{ulem}
\usepackage{makecell}
\usepackage{xcolor}
\makeatletter
\makeatother
\makeatletter
\makeatother
\makeatletter
\@ifpackageloaded{caption}{}{\usepackage{caption}}
\AtBeginDocument{%
\ifdefined\contentsname
  \renewcommand*\contentsname{Table of contents}
\else
  \newcommand\contentsname{Table of contents}
\fi
\ifdefined\listfigurename
  \renewcommand*\listfigurename{List of Figures}
\else
  \newcommand\listfigurename{List of Figures}
\fi
\ifdefined\listtablename
  \renewcommand*\listtablename{List of Tables}
\else
  \newcommand\listtablename{List of Tables}
\fi
\ifdefined\figurename
  \renewcommand*\figurename{Figure}
\else
  \newcommand\figurename{Figure}
\fi
\ifdefined\tablename
  \renewcommand*\tablename{Table}
\else
  \newcommand\tablename{Table}
\fi
}
\@ifpackageloaded{float}{}{\usepackage{float}}
\floatstyle{ruled}
\@ifundefined{c@chapter}{\newfloat{codelisting}{h}{lop}}{\newfloat{codelisting}{h}{lop}[chapter]}
\floatname{codelisting}{Listing}
\newcommand*\listoflistings{\listof{codelisting}{List of Listings}}
\makeatother
\makeatletter
\@ifpackageloaded{caption}{}{\usepackage{caption}}
\@ifpackageloaded{subcaption}{}{\usepackage{subcaption}}
\makeatother
\makeatletter
\@ifpackageloaded{tcolorbox}{}{\usepackage[skins,breakable]{tcolorbox}}
\makeatother
\makeatletter
\@ifundefined{shadecolor}{\definecolor{shadecolor}{rgb}{.97, .97, .97}}
\makeatother
\makeatletter
\makeatother
\makeatletter
\makeatother
\ifLuaTeX
  \usepackage{selnolig}  % disable illegal ligatures
\fi
\IfFileExists{bookmark.sty}{\usepackage{bookmark}}{\usepackage{hyperref}}
\IfFileExists{xurl.sty}{\usepackage{xurl}}{} % add URL line breaks if available
\urlstyle{same} % disable monospaced font for URLs
\hypersetup{
  pdftitle={Spatial correlation},
  pdfauthor={Hans Martinez},
  colorlinks=true,
  linkcolor={blue},
  filecolor={Maroon},
  citecolor={Blue},
  urlcolor={Blue},
  pdfcreator={LaTeX via pandoc}}

\title{Spatial correlation}
\author{Hans Martinez}
\date{2023-08-21}

\begin{document}
\maketitle
\ifdefined\Shaded\renewenvironment{Shaded}{\begin{tcolorbox}[sharp corners, interior hidden, frame hidden, borderline west={3pt}{0pt}{shadecolor}, breakable, boxrule=0pt, enhanced]}{\end{tcolorbox}}\fi

\hypertarget{updates}{%
\section*{Updates}\label{updates}}
\addcontentsline{toc}{section}{Updates}

\begin{itemize}
\tightlist
\item
  Uniformly distributed locations: Table~\ref{tbl-kernel} updated.
  Damian's BIC and Kernel distance. SCPC and C-SCPC rejection
  frequencies. Splines of high order. Average number of dropped splines.
\item
  Evenly distributed locations: Table~\ref{tbl-grid-all} updated.
  Damian's BIC and Kernel distance.
\end{itemize}

\hypertarget{to-do}{%
\section*{To-Do}\label{to-do}}
\addcontentsline{toc}{section}{To-Do}

\begin{itemize}
\tightlist
\item
  Evenly distributed locations: SCPC adjusting \(\bar\rho\) to avoid
  crashing.
\item
  Power curves for SCPC method
\item
  Cluster version of Kernel estimator
\item
  Measurement error in the locations

  \begin{enumerate}
  \def\labelenumi{\alph{enumi}.}
  \tightlist
  \item
    Get rejection freq.
  \item
    Plot distances (true vs.~mismeasured)
  \end{enumerate}
\end{itemize}

\hypertarget{sec-kernel}{%
\section{Kernel estimator}\label{sec-kernel}}

I compare the results for the four different variance estimators, the
HR, SCPC, C-SCPC, and Kernel. For the Kernel estimator, I try different
cut-off distances. In addition, B-Splines of different order and
quantities are introduced to help improve the rejection frequency of the
Kernel estimator.

I'm using one and two-order B-Splines (step functions and triangles).
During the estimation, a spline was dropped if it contained one
observation or less.

In the DGP, \(y=\beta_0 + \beta_1 x_l +e_l\), where
\(e_l\sim\mathcal{G}_{exp}(c_{\bar\rho})\) and
\(X\sim[1 \;\mathcal{G}_{exp}(c_{\bar\rho})]\). The true parameters are
\(\beta=(1, 1.5)'\), and the average pairwise correlation is
\(\bar\rho=0.03\).

For every ols regression, I estimated an AR(1) model using the residuals
\(\hat{e}_{l+1}=\gamma_0+\gamma_1 \hat{e}_l + u_l\), after sorting from
closest to zero to closest to one.

I used 1000 iterations and 250 observations to calculate the rejection
frequencies and the average AR(1) coefficients of the residuals.

Table~\ref{tbl-kernel} displays the rejection frequencies for the
different estimators, the average AR(1) slope coefficient of the
residuals \(\gamma_1\), the average BIC for two methods, and the average
number of splines dropped.

\hypertarget{bic}{%
\subsection{BIC}\label{bic}}

Hansen's BIC is estimated as
\(BIC_H=T\log(2\pi\hat{\sigma}^2_{e_l})+T+(k+1)\log(T)\) (Hansen 2020),
while Damian's is estimated
\(BIC_D=T\log(\hat{\sigma}^2_{e_l})+k\log(T)\)

\hypertarget{tbl-kernel}{}
\begin{longtable}[t]{lllrrrrrr}
\caption{\label{tbl-kernel}Rejection frequencies for different variance estimators. Spatial data.
250 observations. 1000 iterations. }\tabularnewline

\toprule
\multicolumn{5}{c}{ } & \multicolumn{4}{c}{Residuals} \\
\cmidrule(l{3pt}r{3pt}){6-9}
\multicolumn{2}{c}{B-Splines} & \multicolumn{3}{c}{Spatial data} & \multicolumn{1}{c}{ } & \multicolumn{2}{c}{BIC} \\
\cmidrule(l{3pt}r{3pt}){1-2} \cmidrule(l{3pt}r{3pt}){3-5} \cmidrule(l{3pt}r{3pt}){7-8}
Order & Qty. & Estimator & Cons. & $\beta_1$ & $\gamma_1$ & Hansen's & Damian's & Dropped\\
\midrule \endhead
 &  & HR & 0.498 & 0.385 &  &  &  & \\

 &  & SCPC & 0.067 & 0.073 &  &  &  & \\

 &  & C-SCPC & 0.055 & 0.054 &  &  &  & \\

 &  & Kernel-0.015 & 0.138 & 0.107 &  &  &  & \\

 &  & Kernel-0.030 & 0.101 & 0.117 &  &  &  & \\

\multirow[t]{-6}{*}{\raggedright\arraybackslash } & \multirow[t]{-6}{*}{\raggedright\arraybackslash No splines} & Kernel-0.050 & 0.114 & 0.135 & \multirow[t]{-6}{*}{\raggedleft\arraybackslash 0.770} & \multirow[t]{-6}{*}{\raggedleft\arraybackslash 708.622} & \multirow[t]{-6}{*}{\raggedleft\arraybackslash -6.368} & \multirow[t]{-6}{*}{\raggedleft\arraybackslash }\\
\cmidrule{1-9}
\multirow[t]{45}{*}{\raggedright\arraybackslash Order 1} &  & HR &  & 0.428 &  &  &  & \multirow[t]{20}{*}{\raggedleft\arraybackslash 0}\\

 &  & Kernel-0.015 &  & 0.129 &  &  &  & \\

 &  & Kernel-0.030 &  & 0.111 &  &  &  & \\

 &  & Kernel-0.050 &  & 0.128 &  &  &  & \\

 & \multirow[t]{-5}{*}{\raggedright\arraybackslash 4 splines} & Damian's &  & 1.000 & \multirow[t]{-5}{*}{\raggedleft\arraybackslash 0.805} & \multirow[t]{-5}{*}{\raggedleft\arraybackslash 761.473} & \multirow[t]{-5}{*}{\raggedleft\arraybackslash 46.483} & \\
\cmidrule{2-8}
 &  & HR &  & 0.388 &  &  &  & \\

 &  & Kernel-0.015 &  & 0.135 &  &  &  & \\

 &  & Kernel-0.030 &  & 0.136 &  &  &  & \\

 &  & Kernel-0.050 &  & 0.155 &  &  &  & \\

 & \multirow[t]{-5}{*}{\raggedright\arraybackslash 12 splines} & Damian's &  & 0.295 & \multirow[t]{-5}{*}{\raggedleft\arraybackslash 0.730} & \multirow[t]{-5}{*}{\raggedleft\arraybackslash 726.707} & \multirow[t]{-5}{*}{\raggedleft\arraybackslash 11.716} & \\
\cmidrule{2-8}
 &  & HR &  & 0.339 &  &  &  & \\

 &  & Kernel-0.015 &  & 0.132 &  &  &  & \\

 &  & Kernel-0.030 &  & 0.105 &  &  &  & \\

 &  & Kernel-0.050 &  & 0.097 &  &  &  & \\

 & \multirow[t]{-5}{*}{\raggedright\arraybackslash 24 splines} & Damian's &  & 0.175 & \multirow[t]{-5}{*}{\raggedleft\arraybackslash 0.588} & \multirow[t]{-5}{*}{\raggedleft\arraybackslash 680.585} & \multirow[t]{-5}{*}{\raggedleft\arraybackslash -34.406} & \\
\cmidrule{2-8} \pagebreak
 &  & HR &  & 0.269 &  &  &  & \\

 &  & Kernel-0.015 &  & 0.100 &  &  &  & \\

 &  & Kernel-0.030 &  & 0.088 &  &  &  & \\

 &  & Kernel-0.050 &  & 0.102 &  &  &  & \\

 & \multirow[t]{-5}{*}{\raggedright\arraybackslash 36 splines} & Damian's &  & 0.135 & \multirow[t]{-5}{*}{\raggedleft\arraybackslash 0.469} & \multirow[t]{-5}{*}{\raggedleft\arraybackslash 671.021} & \multirow[t]{-5}{*}{\raggedleft\arraybackslash -43.970} & \\
\cmidrule{2-9}
 &  & HR &  & 0.231 &  &  &  & \\

 &  & Kernel-0.015 &  & 0.085 &  &  &  & \\

 &  & Kernel-0.030 &  & 0.073 &  &  &  & \\

 &  & Kernel-0.050 &  & 0.094 &  &  &  & \\

 & \multirow[t]{-5}{*}{\raggedright\arraybackslash 48 splines} & Damian's &  & 0.110 & \multirow[t]{-5}{*}{\raggedleft\arraybackslash 0.382} & \multirow[t]{-5}{*}{\raggedleft\arraybackslash 666.330} & \multirow[t]{-5}{*}{\raggedleft\arraybackslash -48.661} & \multirow[t]{-5}{*}{\raggedleft\arraybackslash 3}\\
\cmidrule{2-9}
 &  & HR &  & 0.148 &  &  &  & \\

 &  & Kernel-0.015 &  & 0.056 &  &  &  & \\

 &  & Kernel-0.030 &  & 0.071 &  &  &  & \\

 &  & Kernel-0.050 &  & 0.086 &  &  &  & \\

 & \multirow[t]{-5}{*}{\raggedright\arraybackslash 64 splines} & Damian's &  & 0.085 & \multirow[t]{-5}{*}{\raggedleft\arraybackslash 0.208} & \multirow[t]{-5}{*}{\raggedleft\arraybackslash 677.826} & \multirow[t]{-5}{*}{\raggedleft\arraybackslash -37.165} & \multirow[t]{-5}{*}{\raggedleft\arraybackslash 6}\\
\cmidrule{2-9}
 &  & HR &  & 0.122 &  &  &  & \\

 &  & Kernel-0.015 &  & 0.055 &  &  &  & \\

 &  & Kernel-0.030 &  & 0.056 &  &  &  & \\

 &  & Kernel-0.050 &  & 0.071 &  &  &  & \\

 & \multirow[t]{-5}{*}{\raggedright\arraybackslash 88 splines} & Damian's &  & 0.060 & \multirow[t]{-5}{*}{\raggedleft\arraybackslash 0.172} & \multirow[t]{-5}{*}{\raggedleft\arraybackslash 736.732} & \multirow[t]{-5}{*}{\raggedleft\arraybackslash 21.741} & \multirow[t]{-5}{*}{\raggedleft\arraybackslash 20}\\
\cmidrule{2-9}
 &  & HR &  & 0.102 &  &  &  & \\

 &  & Kernel-0.015 &  & 0.053 &  &  &  & \\

 &  & Kernel-0.030 &  & 0.064 &  &  &  & \\

 &  & Kernel-0.050 &  & 0.082 &  &  &  & \\

 & \multirow[t]{-5}{*}{\raggedright\arraybackslash 102 splines} & Damian's &  & 0.064 & \multirow[t]{-5}{*}{\raggedleft\arraybackslash 0.152} & \multirow[t]{-5}{*}{\raggedleft\arraybackslash 773.159} & \multirow[t]{-5}{*}{\raggedleft\arraybackslash 58.168} & \multirow[t]{-5}{*}{\raggedleft\arraybackslash 31}\\
\cmidrule{2-9}
 &  & HR &  & 0.117 &  &  &  & \\

 &  & Kernel-0.015 &  & 0.060 &  &  &  & \\

 &  & Kernel-0.030 &  & 0.062 &  &  &  & \\

 &  & Kernel-0.050 &  & 0.078 &  &  &  & \\

 & \multirow[t]{-5}{*}{\raggedright\arraybackslash 124 splines} & Damian's &  & 0.059 & \multirow[t]{-5}{*}{\raggedleft\arraybackslash 0.212} & \multirow[t]{-5}{*}{\raggedleft\arraybackslash 841.395} & \multirow[t]{-5}{*}{\raggedleft\arraybackslash 126.404} & \multirow[t]{-5}{*}{\raggedleft\arraybackslash 51}\\
\cmidrule{1-9}
\multirow[t]{45}{*}{\raggedright\arraybackslash Order 2} &  & HR &  & 0.351 &  &  &  & \multirow[t]{30}{*}{\raggedleft\arraybackslash 0}\\

 &  & Kernel-0.015 &  & 0.118 &  &  &  & \\

 &  & Kernel-0.030 &  & 0.129 &  &  &  & \\

 &  & Kernel-0.050 &  & 0.152 &  &  &  & \\

 & \multirow[t]{-5}{*}{\raggedright\arraybackslash 4 splines} & Damian's &  & 1.000 & \multirow[t]{-5}{*}{\raggedleft\arraybackslash 0.747} & \multirow[t]{-5}{*}{\raggedleft\arraybackslash 699.310} & \multirow[t]{-5}{*}{\raggedleft\arraybackslash -15.681} & \\
\cmidrule{2-8}
 &  & HR &  & 0.348 &  &  &  & \\

 &  & Kernel-0.015 &  & 0.138 &  &  &  & \\

 &  & Kernel-0.030 &  & 0.141 &  &  &  & \\

 &  & Kernel-0.050 &  & 0.136 &  &  &  & \\

 & \multirow[t]{-5}{*}{\raggedright\arraybackslash 12 splines} & Damian's &  & 0.292 & \multirow[t]{-5}{*}{\raggedleft\arraybackslash 0.668} & \multirow[t]{-5}{*}{\raggedleft\arraybackslash 671.336} & \multirow[t]{-5}{*}{\raggedleft\arraybackslash -43.654} & \\
\cmidrule{2-8}
 &  & HR &  & 0.316 &  &  &  & \\

 &  & Kernel-0.015 &  & 0.163 &  &  &  & \\

 &  & Kernel-0.030 &  & 0.127 &  &  &  & \\

 &  & Kernel-0.050 &  & 0.124 &  &  &  & \\

 & \multirow[t]{-5}{*}{\raggedright\arraybackslash 24 splines} & Damian's &  & 0.191 & \multirow[t]{-5}{*}{\raggedleft\arraybackslash 0.532} & \multirow[t]{-5}{*}{\raggedleft\arraybackslash 638.916} & \multirow[t]{-5}{*}{\raggedleft\arraybackslash -76.075} & \\
\cmidrule{2-8}
 &  & HR &  & 0.271 &  &  &  & \\

 &  & Kernel-0.015 &  & 0.133 &  &  &  & \\

 &  & Kernel-0.030 &  & 0.098 &  &  &  & \\

 &  & Kernel-0.050 &  & 0.107 &  &  &  & \\

 & \multirow[t]{-5}{*}{\raggedright\arraybackslash 36 splines} & Damian's &  & 0.147 & \multirow[t]{-5}{*}{\raggedleft\arraybackslash 0.392} & \multirow[t]{-5}{*}{\raggedleft\arraybackslash 615.293} & \multirow[t]{-5}{*}{\raggedleft\arraybackslash -99.698} & \\
\cmidrule{2-8}
 &  & HR &  & 0.221 &  &  &  & \\

 &  & Kernel-0.015 &  & 0.082 &  &  &  & \\

 &  & Kernel-0.030 &  & 0.067 &  &  &  & \\

 &  & Kernel-0.050 &  & 0.081 &  &  &  & \\

 & \multirow[t]{-5}{*}{\raggedright\arraybackslash 48 splines} & Damian's &  & 0.086 & \multirow[t]{-5}{*}{\raggedleft\arraybackslash 0.252} & \multirow[t]{-5}{*}{\raggedleft\arraybackslash 600.864} & \multirow[t]{-5}{*}{\raggedleft\arraybackslash -114.127} & \\
\cmidrule{2-8}
 &  & HR &  & 0.165 &  &  &  & \\

 &  & Kernel-0.015 &  & 0.066 &  &  &  & \\

 &  & Kernel-0.030 &  & 0.065 &  &  &  & \\

 &  & Kernel-0.050 &  & 0.077 &  &  &  & \\

 & \multirow[t]{-5}{*}{\raggedright\arraybackslash 64 splines} & Damian's &  & 0.073 & \multirow[t]{-5}{*}{\raggedleft\arraybackslash 0.107} & \multirow[t]{-5}{*}{\raggedleft\arraybackslash 601.697} & \multirow[t]{-5}{*}{\raggedleft\arraybackslash -113.294} & \\
\cmidrule{2-9}
 &  & HR &  & 0.137 &  &  &  & \\

 &  & Kernel-0.015 &  & 0.036 &  &  &  & \\

 &  & Kernel-0.030 &  & 0.041 &  &  &  & \\

 &  & Kernel-0.050 &  & 0.055 &  &  &  & \\

 & \multirow[t]{-5}{*}{\raggedright\arraybackslash 88 splines} & Damian's &  & 0.046 & \multirow[t]{-5}{*}{\raggedleft\arraybackslash -0.054} & \multirow[t]{-5}{*}{\raggedleft\arraybackslash 613.701} & \multirow[t]{-5}{*}{\raggedleft\arraybackslash -101.289} & \multirow[t]{-5}{*}{\raggedleft\arraybackslash 6}\\
\cmidrule{2-9}
 &  & HR &  & 0.085 &  &  &  & \\

 &  & Kernel-0.015 &  & 0.036 &  &  &  & \\

 &  & Kernel-0.030 &  & 0.037 &  &  &  & \\

 &  & Kernel-0.050 &  & 0.048 &  &  &  & \\

 & \multirow[t]{-5}{*}{\raggedright\arraybackslash 102 splines} & Damian's &  & 0.037 & \multirow[t]{-5}{*}{\raggedleft\arraybackslash -0.118} & \multirow[t]{-5}{*}{\raggedleft\arraybackslash 656.343} & \multirow[t]{-5}{*}{\raggedleft\arraybackslash -58.648} & \multirow[t]{-5}{*}{\raggedleft\arraybackslash 13}\\
\cmidrule{2-9}
 &  & HR &  & 0.086 &  &  &  & \\

 &  & Kernel-0.015 &  & 0.045 &  &  &  & \\

 &  & Kernel-0.030 &  & 0.053 &  &  &  & \\

 &  & Kernel-0.050 &  & 0.072 &  &  &  & \\

 & \multirow[t]{-5}{*}{\raggedright\arraybackslash 124 splines} & Damian's &  & 0.050 & \multirow[t]{-5}{*}{\raggedleft\arraybackslash -0.055} & \multirow[t]{-5}{*}{\raggedleft\arraybackslash 718.384} & \multirow[t]{-5}{*}{\raggedleft\arraybackslash 3.393} & \multirow[t]{-5}{*}{\raggedleft\arraybackslash 28}\\
\bottomrule
\end{longtable}

\hypertarget{evenly-spaced-locations-time-series}{%
\section{Evenly spaced locations (time
series)}\label{evenly-spaced-locations-time-series}}

In this exercise, locations are evenly spaced from \(\frac{1}{250}\) to
1. The error terms and the regressors are autocorrelated following an
AR(1) process \(e_{l+1}=\gamma e_l+ u_l\), where \(u_l\sim N(0,1)\), and
\(\gamma=0.79\). Likewise, \(x_{l+1}=\lambda x_l+ v_l\), where
\(v_l\sim N(0,1)\), and \(\lambda=0.79\). The DGP for y is therefore,
\(y_l=\beta_0+\beta_1 x_l+e_l\) with \((\beta_0, \beta_1)=(1, 1.5)'\).

Table~\ref{tbl-grid-all} displays the rejection frequencies of 1000
simulations. SCPC is omitted because it could not find a reasonable
value of \(c_{min}\) for the default value of \(\bar\rho=0.03\) with the
evenly distributed locations. In addition, the table also displays
\(\gamma_1\), the average slope coefficient of an AR(1) model using the
residuals of the OLS regression for every iteration, and the BIC for two
methods as described in Section~\ref{sec-kernel}.

\hypertarget{tbl-grid-all}{}
\begin{longtable}[t]{lllrrrrr}
\caption{\label{tbl-grid-all}Rejection frequencies estimators using time series data. }\tabularnewline

\toprule
\multicolumn{5}{c}{ } & \multicolumn{3}{c}{Residuals} \\
\cmidrule(l{3pt}r{3pt}){6-8}
\multicolumn{2}{c}{B-Splines} & \multicolumn{3}{c}{T.S. data} & \multicolumn{1}{c}{ } & \multicolumn{2}{c}{BIC} \\
\cmidrule(l{3pt}r{3pt}){1-2} \cmidrule(l{3pt}r{3pt}){3-5} \cmidrule(l{3pt}r{3pt}){7-8}
Order & Qty. & Estimator & Cons. & $\beta_1$ & $\gamma$ & Hansen's & Damian's\\
\midrule \endhead
 &  & HR & 0.514 & 0.352 &  &  & \\

 &  & Kernel-0.015 & 0.157 & 0.089 &  &  & \\

 &  & Kernel-0.030 & 0.108 & 0.079 &  &  & \\

\multirow[t]{-4}{*}{\raggedright\arraybackslash } & \multirow[t]{-4}{*}{\raggedright\arraybackslash No splines} & Kernel-0.050 & 0.103 & 0.104 & \multirow[t]{-4}{*}{\raggedleft\arraybackslash 0.769} & \multirow[t]{-4}{*}{\raggedleft\arraybackslash 954.080} & \multirow[t]{-4}{*}{\raggedleft\arraybackslash 239.089}\\
\cmidrule{1-8}
\multirow[t]{45}{*}{\raggedright\arraybackslash Order 1} &  & HR &  & 0.349 &  &  & \\

 &  & Kernel-0.015 &  & 0.088 &  &  & \\

 &  & Kernel-0.030 &  & 0.075 &  &  & \\

 &  & Kernel-0.050 &  & 0.090 &  &  & \\

 & \multirow[t]{-5}{*}{\raggedright\arraybackslash 4 splines} & Kernel-Damian's &  & 1.000 & \multirow[t]{-5}{*}{\raggedleft\arraybackslash 0.776} & \multirow[t]{-5}{*}{\raggedleft\arraybackslash 971.112} & \multirow[t]{-5}{*}{\raggedleft\arraybackslash 256.121}\\
\cmidrule{2-8} \pagebreak
 &  & HR &  & 0.317 &  &  & \\

 &  & Kernel-0.015 &  & 0.120 &  &  & \\

 &  & Kernel-0.030 &  & 0.125 &  &  & \\

 &  & Kernel-0.050 &  & 0.114 &  &  & \\

 & \multirow[t]{-5}{*}{\raggedright\arraybackslash 12 splines} & Kernel-Damian's &  & 0.279 & \multirow[t]{-5}{*}{\raggedleft\arraybackslash 0.692} & \multirow[t]{-5}{*}{\raggedleft\arraybackslash 942.313} & \multirow[t]{-5}{*}{\raggedleft\arraybackslash 227.323}\\
\cmidrule{2-8}
 &  & HR &  & 0.263 &  &  & \\

 &  & Kernel-0.015 &  & 0.121 &  &  & \\

 &  & Kernel-0.030 &  & 0.084 &  &  & \\

 &  & Kernel-0.050 &  & 0.080 &  &  & \\

 & \multirow[t]{-5}{*}{\raggedright\arraybackslash 24 splines} & Kernel-Damian's &  & 0.167 & \multirow[t]{-5}{*}{\raggedleft\arraybackslash 0.546} & \multirow[t]{-5}{*}{\raggedleft\arraybackslash 922.166} & \multirow[t]{-5}{*}{\raggedleft\arraybackslash 207.175}\\
\cmidrule{2-8}
 &  & HR &  & 0.182 &  &  & \\

 &  & Kernel-0.015 &  & 0.095 &  &  & \\

 &  & Kernel-0.030 &  & 0.069 &  &  & \\

 &  & Kernel-0.050 &  & 0.077 &  &  & \\

 & \multirow[t]{-5}{*}{\raggedright\arraybackslash 36 splines} & Kernel-Damian's &  & 0.105 & \multirow[t]{-5}{*}{\raggedleft\arraybackslash 0.404} & \multirow[t]{-5}{*}{\raggedleft\arraybackslash 923.559} & \multirow[t]{-5}{*}{\raggedleft\arraybackslash 208.568}\\
\cmidrule{2-8}
 &  & HR &  & 0.149 &  &  & \\

 &  & Kernel-0.015 &  & 0.070 &  &  & \\

 &  & Kernel-0.030 &  & 0.057 &  &  & \\

 &  & Kernel-0.050 &  & 0.078 &  &  & \\

 & \multirow[t]{-5}{*}{\raggedright\arraybackslash 48 splines} & Kernel-Damian's &  & 0.086 & \multirow[t]{-5}{*}{\raggedleft\arraybackslash 0.269} & \multirow[t]{-5}{*}{\raggedleft\arraybackslash 936.482} & \multirow[t]{-5}{*}{\raggedleft\arraybackslash 221.491}\\
\cmidrule{2-8}
 &  & HR &  & 0.089 &  &  & \\

 &  & Kernel-0.015 &  & 0.035 &  &  & \\

 &  & Kernel-0.030 &  & 0.041 &  &  & \\

 &  & Kernel-0.050 &  & 0.053 &  &  & \\

 & \multirow[t]{-5}{*}{\raggedright\arraybackslash 64 splines} & Kernel-Damian's &  & 0.054 & \multirow[t]{-5}{*}{\raggedleft\arraybackslash 0.088} & \multirow[t]{-5}{*}{\raggedleft\arraybackslash 960.101} & \multirow[t]{-5}{*}{\raggedleft\arraybackslash 245.110}\\
\cmidrule{2-8}
 &  & HR &  & 0.080 &  &  & \\

 &  & Kernel-0.015 &  & 0.032 &  &  & \\

 &  & Kernel-0.030 &  & 0.035 &  &  & \\

 &  & Kernel-0.050 &  & 0.058 &  &  & \\

 & \multirow[t]{-5}{*}{\raggedright\arraybackslash 88 splines} & Kernel-Damian's &  & 0.037 & \multirow[t]{-5}{*}{\raggedleft\arraybackslash -0.130} & \multirow[t]{-5}{*}{\raggedleft\arraybackslash 1007.323} & \multirow[t]{-5}{*}{\raggedleft\arraybackslash 292.332}\\
\cmidrule{2-8}
 &  & HR &  & 0.070 &  &  & \\

 &  & Kernel-0.015 &  & 0.017 &  &  & \\

 &  & Kernel-0.030 &  & 0.029 &  &  & \\

 &  & Kernel-0.050 &  & 0.040 &  &  & \\

 & \multirow[t]{-5}{*}{\raggedright\arraybackslash 102 splines} & Kernel-Damian's &  & 0.029 & \multirow[t]{-5}{*}{\raggedleft\arraybackslash -0.204} & \multirow[t]{-5}{*}{\raggedleft\arraybackslash 1055.792} & \multirow[t]{-5}{*}{\raggedleft\arraybackslash 340.801}\\
\cmidrule{2-8} \pagebreak
 &  & HR &  & 0.052 &  &  & \\

 &  & Kernel-0.015 &  & 0.014 &  &  & \\

 &  & Kernel-0.030 &  & 0.017 &  &  & \\

 &  & Kernel-0.050 &  & 0.027 &  &  & \\

 & \multirow[t]{-5}{*}{\raggedright\arraybackslash 124 splines} & Kernel-Damian's &  & 0.018 & \multirow[t]{-5}{*}{\raggedleft\arraybackslash -0.368} & \multirow[t]{-5}{*}{\raggedleft\arraybackslash 1108.835} & \multirow[t]{-5}{*}{\raggedleft\arraybackslash 393.845}\\
\cmidrule{1-8}
\multirow[t]{45}{*}{\raggedright\arraybackslash Order 2} &  & HR &  & 0.337 &  &  & \\

 &  & Kernel-0.015 &  & 0.100 &  &  & \\

 &  & Kernel-0.030 &  & 0.105 &  &  & \\

 &  & Kernel-0.050 &  & 0.124 &  &  & \\

 & \multirow[t]{-5}{*}{\raggedright\arraybackslash 4 splines} & Kernel-Damian's &  & 1.000 & \multirow[t]{-5}{*}{\raggedleft\arraybackslash 0.744} & \multirow[t]{-5}{*}{\raggedleft\arraybackslash 944.787} & \multirow[t]{-5}{*}{\raggedleft\arraybackslash 229.796}\\
\cmidrule{2-8}
 &  & HR &  & 0.299 &  &  & \\

 &  & Kernel-0.015 &  & 0.141 &  &  & \\

 &  & Kernel-0.030 &  & 0.151 &  &  & \\

 &  & Kernel-0.050 &  & 0.133 &  &  & \\

 & \multirow[t]{-5}{*}{\raggedright\arraybackslash 12 splines} & Kernel-Damian's &  & 0.273 & \multirow[t]{-5}{*}{\raggedleft\arraybackslash 0.659} & \multirow[t]{-5}{*}{\raggedleft\arraybackslash 914.959} & \multirow[t]{-5}{*}{\raggedleft\arraybackslash 199.969}\\
\cmidrule{2-8}
 &  & HR &  & 0.244 &  &  & \\

 &  & Kernel-0.015 &  & 0.138 &  &  & \\

 &  & Kernel-0.030 &  & 0.095 &  &  & \\

 &  & Kernel-0.050 &  & 0.104 &  &  & \\

 & \multirow[t]{-5}{*}{\raggedright\arraybackslash 24 splines} & Kernel-Damian's &  & 0.175 & \multirow[t]{-5}{*}{\raggedleft\arraybackslash 0.515} & \multirow[t]{-5}{*}{\raggedleft\arraybackslash 886.166} & \multirow[t]{-5}{*}{\raggedleft\arraybackslash 171.175}\\
\cmidrule{2-8}
 &  & HR &  & 0.187 &  &  & \\

 &  & Kernel-0.015 &  & 0.130 &  &  & \\

 &  & Kernel-0.030 &  & 0.084 &  &  & \\

 &  & Kernel-0.050 &  & 0.089 &  &  & \\

 & \multirow[t]{-5}{*}{\raggedright\arraybackslash 36 splines} & Kernel-Damian's &  & 0.135 & \multirow[t]{-5}{*}{\raggedleft\arraybackslash 0.359} & \multirow[t]{-5}{*}{\raggedleft\arraybackslash 871.928} & \multirow[t]{-5}{*}{\raggedleft\arraybackslash 156.937}\\
\cmidrule{2-8}
 &  & HR &  & 0.161 &  &  & \\

 &  & Kernel-0.015 &  & 0.089 &  &  & \\

 &  & Kernel-0.030 &  & 0.076 &  &  & \\

 &  & Kernel-0.050 &  & 0.082 &  &  & \\

 & \multirow[t]{-5}{*}{\raggedright\arraybackslash 48 splines} & Kernel-Damian's &  & 0.092 & \multirow[t]{-5}{*}{\raggedleft\arraybackslash 0.207} & \multirow[t]{-5}{*}{\raggedleft\arraybackslash 872.475} & \multirow[t]{-5}{*}{\raggedleft\arraybackslash 157.484}\\
\cmidrule{2-8}
 &  & HR &  & 0.110 &  &  & \\

 &  & Kernel-0.015 &  & 0.056 &  &  & \\

 &  & Kernel-0.030 &  & 0.050 &  &  & \\

 &  & Kernel-0.050 &  & 0.056 &  &  & \\

 & \multirow[t]{-5}{*}{\raggedright\arraybackslash 64 splines} & Kernel-Damian's &  & 0.058 & \multirow[t]{-5}{*}{\raggedleft\arraybackslash 0.024} & \multirow[t]{-5}{*}{\raggedleft\arraybackslash 890.449} & \multirow[t]{-5}{*}{\raggedleft\arraybackslash 175.458}\\
\cmidrule{2-8} \pagebreak
 &  & HR &  & 0.078 &  &  & \\

 &  & Kernel-0.015 &  & 0.044 &  &  & \\

 &  & Kernel-0.030 &  & 0.041 &  &  & \\

 &  & Kernel-0.050 &  & 0.055 &  &  & \\

 & \multirow[t]{-5}{*}{\raggedright\arraybackslash 88 splines} & Kernel-Damian's &  & 0.043 & \multirow[t]{-5}{*}{\raggedleft\arraybackslash -0.219} & \multirow[t]{-5}{*}{\raggedleft\arraybackslash 934.837} & \multirow[t]{-5}{*}{\raggedleft\arraybackslash 219.846}\\
\cmidrule{2-8}
 &  & HR &  & 0.064 &  &  & \\

 &  & Kernel-0.015 &  & 0.022 &  &  & \\

 &  & Kernel-0.030 &  & 0.029 &  &  & \\

 &  & Kernel-0.050 &  & 0.044 &  &  & \\

 & \multirow[t]{-5}{*}{\raggedright\arraybackslash 102 splines} & Kernel-Damian's &  & 0.029 & \multirow[t]{-5}{*}{\raggedleft\arraybackslash -0.343} & \multirow[t]{-5}{*}{\raggedleft\arraybackslash 965.463} & \multirow[t]{-5}{*}{\raggedleft\arraybackslash 250.472}\\
\cmidrule{2-8}
 &  & HR &  & 0.056 &  &  & \\

 &  & Kernel-0.015 &  & 0.012 &  &  & \\

 &  & Kernel-0.030 &  & 0.020 &  &  & \\

 &  & Kernel-0.050 &  & 0.035 &  &  & \\

 & \multirow[t]{-5}{*}{\raggedright\arraybackslash 124 splines} & Kernel-Damian's &  & 0.015 & \multirow[t]{-5}{*}{\raggedleft\arraybackslash -0.508} & \multirow[t]{-5}{*}{\raggedleft\arraybackslash 1018.641} & \multirow[t]{-5}{*}{\raggedleft\arraybackslash 303.650}\\
\bottomrule
\end{longtable}

Table~\ref{tbl-rho-bar-ar1} displays the average pairwise correlation
for groups of different cut-off distances. The average pairwise
correlation was estimated as follows

\[
\bar\rho=(T(T-1))^{-1}\Sigma_{l=1}^T\Sigma_{l\not=\text{l}}^T[e_le_{\text{l}}\mathbb{I}(|s_l-s_{\text{l}}|<L)]
\]

\hypertarget{tbl-rho-bar-ar1}{}
\begin{table}
\caption{\label{tbl-rho-bar-ar1}Average pairwise correlation within observations of groups for different
distances.
\(\bar\rho=(T(T-1))^{-1}\Sigma_{l=1}^T\Sigma_{l\not=\text{l}}^T[e_le_{\text{l}}\mathbb{I}(|s_l-s_{\text{l}}|<L)]\) }\tabularnewline

\centering
\begin{tabular}[t]{rr}
\toprule
Distance & $\bar\rho$\\
\midrule
0.015 & 1.007\\
0.050 & 0.279\\
0.100 & 0.085\\
0.500 & 0.003\\
1.000 & -0.008\\
\bottomrule
\end{tabular}
\end{table}

\hypertarget{locations-and-distances-new}{%
\section{Locations and distances
(NEW)}\label{locations-and-distances-new}}

Figure~\ref{fig-loc} shows the histogram of 250 locations when they are
uniformly distributed and evenly spaced. Figure~\ref{fig-dist} shows the
histogram of the corresponding pairwise distances.
Figure~\ref{fig-hist-loc-x10} shows the same graphs but for 2,500
locations.

As the number of locations increases, the plots of the uniformly
distributed locations look more like the ones that are evenly spaced.

\begin{figure}

\begin{minipage}[t]{\linewidth}

{\centering 

\raisebox{-\height}{

\includegraphics{"../Simulations/Products/hist_loc.png"}

}

}

\subcaption{\label{fig-loc}Locations}
\end{minipage}%
\newline
\begin{minipage}[t]{\linewidth}

{\centering 

\raisebox{-\height}{

\includegraphics{"../Simulations/Products/hist_dist.png"}

}

}

\subcaption{\label{fig-dist}Distances}
\end{minipage}%

\caption{\label{fig-hist-loc}Histograms of uniformly distributed
locations (blue) and evenly-spaced (time-series) locations (red) and
their corresponding pairwise distances. 250 locations.}

\end{figure}

\begin{figure}

\begin{minipage}[t]{\linewidth}

{\centering 

\raisebox{-\height}{

\includegraphics{"../Simulations/Products/hist_loc_2500.png"}

}

}

\subcaption{\label{fig-loc-x10}Locations}
\end{minipage}%
\newline
\begin{minipage}[t]{\linewidth}

{\centering 

\raisebox{-\height}{

\includegraphics{"../Simulations/Products/hist_dist_2500.png"}

}

}

\subcaption{\label{fig-dist-x10}Distances}
\end{minipage}%

\caption{\label{fig-hist-loc-x10}Histograms of uniformly distributed
locations (blue) and evenly-spaced (time-series) locations (red) and
their corresponding pairwise distances. 2,500 locations.}

\end{figure}

\hypertarget{scpc-limitations}{%
\section{SCPC limitations}\label{scpc-limitations}}

Open questions to address in Spatial Correlation Principal Component
(SCPC) (Müller and Watson 2022b, 2022a):

\begin{enumerate}
\def\labelenumi{\arabic{enumi}.}
\tightlist
\item
  What's the ``worst case'' covariance matrix? How does that work? A:
  See section on that Section~\ref{sec-worstcov}
\item
  Can SCPC accommodate a vector of variables? A: Not currently. It only
  works for scalars. The authors discuss that the main challenge for the
  multivariate version of the SCPC is specifying the worst-case
  benchmark model and numerically determining the critical value. In the
  case of the C-SCPC, the challenge is to specify an appropriate
  multivariate extension of the heteroscedastic model foe \(e_l\) given
  \(e_l=x^s_la_l\).
\item
  Does the method work if there exists measurement error in the
  locations?
\end{enumerate}

\hypertarget{dgps-covariance-structure}{%
\section{DGP's covariance structure}\label{dgps-covariance-structure}}

Figure~\ref{fig-dist-cov} to \ref{fig-extras} show graphically the
covariance structure of the DGP, in particular with respect to locations
and distance.

\begin{figure}

{\centering \includegraphics{SCPC_files/figure-pdf/fig-dist-cov-1.pdf}

}

\caption{\label{fig-dist-cov}Distance vs Covariance}

\end{figure}

\begin{figure}

{\centering \includegraphics{SCPC_files/figure-pdf/fig-cov-hist-1.pdf}

}

\caption{\label{fig-cov-hist}Covariance histogram}

\end{figure}

\begin{figure}

\begin{minipage}[t]{0.50\linewidth}

{\centering 

\raisebox{-\height}{

\includegraphics{SCPC_files/figure-pdf/fig-extras-1.pdf}

}

}

\subcaption{\label{fig-extras-1}Distance}
\end{minipage}%
%
\begin{minipage}[t]{0.50\linewidth}

{\centering 

\raisebox{-\height}{

\includegraphics{SCPC_files/figure-pdf/fig-extras-2.pdf}

}

}

\subcaption{\label{fig-extras-2}Locations}
\end{minipage}%

\caption{\label{fig-extras}Distance and location histograms}

\end{figure}

\hypertarget{power-of-the-test-using-scpc}{%
\section{Power of the test using
SCPC}\label{power-of-the-test-using-scpc}}

Müller and Watson (2022b) discuss briefly the power of the tests using
their variance estimator. They discuss the difference in the power of
the tests using the conditional and the unconditional SCPC, the
trade-off in the length of the CI and the number of the principal
components used to calculate \(\sigma_{SCPC}\), and the expected length
of the CI as a function of \(\bar\rho_{max}\).

\hypertarget{conditional-vs.-unconditional-scpc-p.8}{%
\subsection{Conditional vs.~Unconditional SCPC
(p.8)}\label{conditional-vs.-unconditional-scpc-p.8}}

The authors argue that the size-adjusted power (average length of the
confidence intervals) of SCPC and C-SCPC (conditional) are identical
because both methods are based on the same \emph{t}-statistic and only
differ in their critical values. The authors argue that the differences
in the average confidence interval lengths of both methods are small.
They based their argument on an unreported comparison realized for only
one experiment design (p.~8).

The authors acknowledge that the size-adjusted Kernel method is
``somewhat'' more efficient than SCPC and C-SCPC. The authors argue
however that this is not a reason to prefer the Kernel method because in
practice the ``size-adjustment that adjust for the larger bias in the
\(\hat\sigma_{K}^2\)'' is not feasible.

\hypertarget{trade-off-in-choosing-q-p.4}{%
\subsection{Trade-off in choosing q
(p.4)}\label{trade-off-in-choosing-q-p.4}}

The authors discuss that there is a trade-off between the number of
principal components \(q\) and the expected length of the 95\%
confidence intervals. In particular, for a fixed critical value, the
expected length of the confidence interval falls as \(q\) increases, but
larger \(q\) requires larger critical values to control coverage. The
authors suggest that minimizing the expected 95\% confidence interval
length under the iid benchmark yields a value of \(q\) that works well
for a range of values of \(c\).

\hypertarget{expected-length-of-confidence-intervals-as-a-function-of-barrho_max-p.1112}{%
\subsection{\texorpdfstring{Expected length of confidence intervals as a
function of \(\bar\rho_{max}\)
(p.11,12)}{Expected length of confidence intervals as a function of \textbackslash bar\textbackslash rho\_\{max\} (p.11,12)}}\label{expected-length-of-confidence-intervals-as-a-function-of-barrho_max-p.1112}}

The authors also discuss what happens to the rejection frequencies and
confidence intervals if the researcher misjudges the spatial correlation
\(\bar\rho_{max}\). For example, if the researcher provides a
significantly low average pairwise correlation \(\bar\rho_{max}=0.03\)
when the true is actually higher \(\bar\rho=0.10\), the authors argue
that the average rejection increases but marginally. The authors
performed a set of 14 experiments with different DGPs, first with the
true \(\bar\rho=0.03\) and then increased to 0.10, but they kept the
provided \(\bar\rho_{max}=0.03\). They observed that the average
rejection frequency increased from 0.04 to 0.07. However, when you look
at the supplemental materials the rejection frequency for some of the
experiments went up even above 0.4.

The authors also discuss how the confidence intervals increase if the
researcher gives a higher value of \(\bar\rho_{max}\) than the true
\(\bar\rho\). The authors compare the CI generated by the C-SCPC method
to the length of the oracle \(\pm 1.96\) interval length when the
\(\bar\rho_{max}={0.1, 0.03,0.01,0.003}\). The authors find that the
average C-SCPC CI is 1.55, 1.2, 1.1, and 1.05 times larger than the
oracle \(\pm 1.96\) CI. The authors conclude that the cost of the
default value of \(\bar\rho=0.03\) is about a 20\% increase in the CI
lengths.

\hypertarget{summary-of-the-scpc-stata-output}{%
\section{\texorpdfstring{Summary of the \texttt{scpc} stata
output}{Summary of the scpc stata output}}\label{summary-of-the-scpc-stata-output}}

Description of the stata output from the post-estimation command
\texttt{scpc}:

\begin{itemize}
\tightlist
\item
  Coefficient
\item
  S.E.: The \texttt{scpc} post-estimation command displays the standard
  error as the SCPC estimated variance using eq. 5 modified as follows
  \(\hat\sigma_{SCPC}/(\sqrt{n}\sqrt{q})\) (line 559,
  \texttt{scpc.ado}).
\item
  t: The command reports the t-statistic, using eq. 4, but using the
  SCPC variance estimator (eq. 5)
\item
  ``P\textgreater\textbar t\textbar{}'': This value corresponds to the
  maximum rejection probability for the adjusted t-statistic. The
  maximum rejection probability, \(\mathbb{P}_{c\ge c_{max}}(\cdot)\),
  is the maximum of the rejection probabilities computed for a range of
  \(c\) values, that go from \(c_{min}\) to \(c_{max}\). \(c_{min}\) is
  provided by the researcher as a maximum average pairwise correlation,
  \(\bar\rho\). The default is \(\bar\rho=0.03\). \(c_{max}\) is the
  maximum value for which size control is checked. In the code,
  \(c_{max}\) corresponds to a minimum average pairwise correlation of
  \(0.00001\) (line 478, \texttt{scpc.ado}).
\item
  Confidence interval: The \texttt{scpc} command computes the CI using
  the critical value that corresponds to a \(5\%\) maximum rejection
  probability. For the range of \(c\) values \(\in[c_{min},c_{max}]\),
  the algorithm searches for the critical value that corresponds to a
  maximum rejection probability \(\alpha=0.05\), the desred size of the
  test.
\end{itemize}

\hypertarget{computing-scpcs-rejection-probability}{%
\section{Computing SCPC's rejection
probability}\label{computing-scpcs-rejection-probability}}

\(\mathbb{P}_{c\ge c_{max}}(\cdot)\) computes the maximum rejection
probability for a range of \(c\) values that go from \(c_{min}\)
---which corresponds to \(\hat\rho\)--- to \(c_{max}\), a minimum value
for which size control is checked. In the code, this value corresponds
to a minimum average pairwise correlation of \(0.00001\) (line 478,
\texttt{scpc.ado}).

The rejection probability is computed using a Gaussian quadrature and
the eigenvalues of \(\Omega(c)\) (pag. 12), for a given critical value.

Then, the critical value is selected such that the rejection probability
is the desired, for example, \(\alpha=0.05\) (line 144,
\texttt{scpc.ado})

\hypertarget{the-setting}{%
\section{The setting}\label{the-setting}}

\[
y_l = x_l\beta + e_l
\]

\(\mathbb{E}[e_l|x_l]=0\), and \((y_l, x_l, e_l)\) are associated with
observed spatial location \(s_l \in \mathbb{R}^d\). Spatial location in
time is \(d=1\), two dimensions, like in altitude and latitude, \(d=2\),
and so on.

\(e_l\) is generated by a Gaussian process with covariance function
\(cov(e_l,e_{l'})=\exp(-c||s_l-s_{l'}||)\), where \(s_l\) and \(s_{l'}\)
denote the spatial locations of \(e_l\) and \(e_{l'}\), and \(c>0\) is a
parameter that governs the strength of the spatial correlation. The
value of \(c\) is calibrated to induce a specific average pairwise
correlation
\(\bar{\rho}=[n(n-1)]^{-1}\sum_{l,l'\not=l}cov(e_l,e_{l'})\). In other
words, \(c=c_{\bar{\rho}}\) solves
\([n(n-1)]^{-1}\sum_{l,l'\not=l}\exp(-c_{\bar{\rho}}||s_l-s_{l'}||)=\bar{\rho}\).

Let \(\Sigma(c)\) be the covariance matrix of \(e_l\) evaluated at the
sample locations, so that \(\Sigma(c)_{l,l'}=\exp(-c||s_l-s_{l'} ||)\),
and let \(\bar{\rho}(c)\) denote the resulting average pairwise
correlation
\(\bar{\rho}(c)=[n(n-1)]^{-1}\sum_{l,l'\not=l}\Sigma(c)_{l,l'}\). If the
researcher desires a test that controls size for values of
\(\bar{\rho}\) as large as \(\bar{\rho}_{max}\), then he can choose
\(c_{min}\) such that \(\bar{\rho}(c_{min})=\bar{\rho}_{max}\). Then,
\(\Sigma(c_{min})\) is the \textbf{worst case} covariance matrix in the
sense that it induces the largest value of \(\sigma^2\) among all
\(\Sigma(c)\) with \(\bar{\rho}\le \bar{\rho}_{max}\)

\hypertarget{sec-worstcov}{%
\section{SCPC's worst case covariance function}\label{sec-worstcov}}

Müller and Watson (2022b) assume a functional form for the covariance
function of the error term that depends on a single parameter. This
single parameter has a unique value for a specific average pairwise
correlation. The researcher has to provide before the estimation, a
maximum average pairwise correlation. Then, because the average pairwise
correlation is a decreasing function of the parameter of the covariance
function, this parameter induces the largest value of the error variance
for the given average pairwise correlation, according to the authors. In
the STATA and MATLAB codes, this parameter is set to 0.03 in both, like
in their paper. The authors provide instructions on how to adjust the
parameter in the help section of the STATA code. In the MATLAB code, a
parameter can be adjusted before estimations.

\hypertarget{literature-review}{%
\section{Literature Review}\label{literature-review}}

\hypertarget{muller2022jbes}{%
\subsection{Müller and Watson (2022b)}\label{muller2022jbes}}

The objective of the paper is to develop a robustified version of their
Spatial Correlation Principal Components (SCPC) (Müller and Watson
2022a). In particular, the authors propose modifications to deal with
non-stationarities and strong dependence in finite samples (small
samples).

According to the authors, the SPCPC method addresses the challenge of
spatial correlation robust inference under small samples and
\emph{empirically relevant} forms of strong dependence (which ones?).

According to the authors, Conley (1999) doesn't work well in small
samples because it relies on the consistency of the estimator of
\(\sigma^2\), while SCPC (and fixed-b type approaches) rely on the
\emph{stationarity} of \(u_l\). Stationarity might break in practice,
for example, when \(x_l\) is a dummy for treatment, and treatment is
more likely in one region than another region.

The SCPC method is based on a principal component estimator of
\(\sigma^2\) based on a pre-specified ``worst-case'' exponential
covariance function conditional on the observed locations.

(On going\ldots)

\hypertarget{references}{%
\section*{References}\label{references}}
\addcontentsline{toc}{section}{References}

\hypertarget{refs}{}
\begin{CSLReferences}{1}{0}
\leavevmode\vadjust pre{\hypertarget{ref-Conley1999}{}}%
Conley, T G. 1999. {``GMM Estimation with Cross Sectional Dependence.''}
\emph{Journal of Econometrics} 92: 1--45.

\leavevmode\vadjust pre{\hypertarget{ref-Hansen2020}{}}%
Hansen, Bruce E. 2020. \emph{ECONOMETRICS}.

\leavevmode\vadjust pre{\hypertarget{ref-Muller2022ECTA}{}}%
Müller, Ulrich K, and Mark W Watson. 2022a. {``SPATIAL CORRELATION
ROBUST INFERENCE.''} \emph{Econometrica} 90: 2901--35.
\url{https://doi.org/10.3982/ECTA19465}.

\leavevmode\vadjust pre{\hypertarget{ref-Muller2022JBES}{}}%
---------. 2022b. {``Spatial Correlation Robust Inference in Linear
Regression and Panel Models.''} \emph{Journal of Business \& Economic
Statistics} 00: 1--15.
\url{https://doi.org/10.1080/07350015.2022.2127737}.

\end{CSLReferences}



\end{document}
