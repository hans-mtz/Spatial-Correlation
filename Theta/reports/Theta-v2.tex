% Options for packages loaded elsewhere
\PassOptionsToPackage{unicode}{hyperref}
\PassOptionsToPackage{hyphens}{url}
\PassOptionsToPackage{dvipsnames,svgnames,x11names}{xcolor}
%
\documentclass[
]{article}

\usepackage{amsmath,amssymb}
\usepackage{iftex}
\ifPDFTeX
  \usepackage[T1]{fontenc}
  \usepackage[utf8]{inputenc}
  \usepackage{textcomp} % provide euro and other symbols
\else % if luatex or xetex
  \usepackage{unicode-math}
  \defaultfontfeatures{Scale=MatchLowercase}
  \defaultfontfeatures[\rmfamily]{Ligatures=TeX,Scale=1}
\fi
\usepackage{lmodern}
\ifPDFTeX\else  
    % xetex/luatex font selection
\fi
% Use upquote if available, for straight quotes in verbatim environments
\IfFileExists{upquote.sty}{\usepackage{upquote}}{}
\IfFileExists{microtype.sty}{% use microtype if available
  \usepackage[]{microtype}
  \UseMicrotypeSet[protrusion]{basicmath} % disable protrusion for tt fonts
}{}
\makeatletter
\@ifundefined{KOMAClassName}{% if non-KOMA class
  \IfFileExists{parskip.sty}{%
    \usepackage{parskip}
  }{% else
    \setlength{\parindent}{0pt}
    \setlength{\parskip}{6pt plus 2pt minus 1pt}}
}{% if KOMA class
  \KOMAoptions{parskip=half}}
\makeatother
\usepackage{xcolor}
\setlength{\emergencystretch}{3em} % prevent overfull lines
\setcounter{secnumdepth}{5}
% Make \paragraph and \subparagraph free-standing
\ifx\paragraph\undefined\else
  \let\oldparagraph\paragraph
  \renewcommand{\paragraph}[1]{\oldparagraph{#1}\mbox{}}
\fi
\ifx\subparagraph\undefined\else
  \let\oldsubparagraph\subparagraph
  \renewcommand{\subparagraph}[1]{\oldsubparagraph{#1}\mbox{}}
\fi


\providecommand{\tightlist}{%
  \setlength{\itemsep}{0pt}\setlength{\parskip}{0pt}}\usepackage{longtable,booktabs,array}
\usepackage{calc} % for calculating minipage widths
% Correct order of tables after \paragraph or \subparagraph
\usepackage{etoolbox}
\makeatletter
\patchcmd\longtable{\par}{\if@noskipsec\mbox{}\fi\par}{}{}
\makeatother
% Allow footnotes in longtable head/foot
\IfFileExists{footnotehyper.sty}{\usepackage{footnotehyper}}{\usepackage{footnote}}
\makesavenoteenv{longtable}
\usepackage{graphicx}
\makeatletter
\def\maxwidth{\ifdim\Gin@nat@width>\linewidth\linewidth\else\Gin@nat@width\fi}
\def\maxheight{\ifdim\Gin@nat@height>\textheight\textheight\else\Gin@nat@height\fi}
\makeatother
% Scale images if necessary, so that they will not overflow the page
% margins by default, and it is still possible to overwrite the defaults
% using explicit options in \includegraphics[width, height, ...]{}
\setkeys{Gin}{width=\maxwidth,height=\maxheight,keepaspectratio}
% Set default figure placement to htbp
\makeatletter
\def\fps@figure{htbp}
\makeatother

% \usepackage[figon, printfigures]{figcaps}
\usepackage{booktabs}
\usepackage{longtable}
\usepackage{array}
\usepackage{multirow}
\usepackage{wrapfig}
\usepackage{float}
\usepackage{colortbl}
\usepackage{pdflscape}
\usepackage{tabu}
\usepackage{threeparttable}
\usepackage{threeparttablex}
\usepackage[normalem]{ulem}
\usepackage{makecell}
\usepackage{xcolor}
\makeatletter
\@ifpackageloaded{caption}{}{\usepackage{caption}}
\AtBeginDocument{%
\ifdefined\contentsname
  \renewcommand*\contentsname{Table of contents}
\else
  \newcommand\contentsname{Table of contents}
\fi
\ifdefined\listfigurename
  \renewcommand*\listfigurename{List of Figures}
\else
  \newcommand\listfigurename{List of Figures}
\fi
\ifdefined\listtablename
  \renewcommand*\listtablename{List of Tables}
\else
  \newcommand\listtablename{List of Tables}
\fi
\ifdefined\figurename
  \renewcommand*\figurename{Figure}
\else
  \newcommand\figurename{Figure}
\fi
\ifdefined\tablename
  \renewcommand*\tablename{Table}
\else
  \newcommand\tablename{Table}
\fi
}
\@ifpackageloaded{float}{}{\usepackage{float}}
\floatstyle{ruled}
\@ifundefined{c@chapter}{\newfloat{codelisting}{h}{lop}}{\newfloat{codelisting}{h}{lop}[chapter]}
\floatname{codelisting}{Listing}
\newcommand*\listoflistings{\listof{codelisting}{List of Listings}}
\makeatother
\makeatletter
\makeatother
\makeatletter
\@ifpackageloaded{caption}{}{\usepackage{caption}}
\@ifpackageloaded{subcaption}{}{\usepackage{subcaption}}
\makeatother
\ifLuaTeX
  \usepackage{selnolig}  % disable illegal ligatures
\fi
\usepackage{bookmark}

\IfFileExists{xurl.sty}{\usepackage{xurl}}{} % add URL line breaks if available
\urlstyle{same} % disable monospaced font for URLs
\hypersetup{
  pdftitle={Replication - Morgan},
  pdfauthor={Hans Martinez},
  colorlinks=true,
  linkcolor={blue},
  filecolor={Maroon},
  citecolor={Blue},
  urlcolor={Blue},
  pdfcreator={LaTeX via pandoc}}

\title{Replication - Morgan}
\author{Hans Martinez}
\date{2024-07-24}

\begin{document}
\maketitle

\section{Morgan's Code}\label{morgans-code}

\subsection{Intercept vs No Intercept}\label{intercept-vs-no-intercept}

\begin{table}

\caption{\label{tbl-img-intercept-u-any}Share of any imaginary SE in
simulations for HAC+Bsplines when including the intercept in OLS.
Uniform Kernel HAC. Morgan's Locations and Splines.}

\centering{

\begin{tabular}[t]{rrrrrrrrrr}
\toprule
\multicolumn{10}{c}{Any} \\
\cmidrule(l{3pt}r{3pt}){1-10}
\multicolumn{2}{c}{ } & \multicolumn{4}{c}{Unif-HAC+Morgan Splines} \\
\cmidrule(l{3pt}r{3pt}){3-6}
...rho. & corr & 0.05 & 0.1 & 0.15 & 0.2 & NN & Drop & HR & BIC\\
\midrule
0.0 & 0.000 & 0.385 & 1.000 & 1 & 1 & -0.106 & 0 & 0.040 & 1759.516\\
0.2 & 0.099 & 0.375 & 1.000 & 1 & 1 & -0.065 & 0 & 0.050 & 1682.078\\
0.4 & 0.197 & 0.330 & 1.000 & 1 & 1 & -0.017 & 0 & 0.030 & 1588.460\\
0.6 & 0.296 & 0.215 & 0.990 & 1 & 1 & 0.022 & 0 & 0.070 & 1478.789\\
0.7 & 0.345 & 0.160 & 0.960 & 1 & 1 & 0.049 & 0 & 0.100 & 1408.481\\
\addlinespace
0.8 & 0.394 & 0.095 & 0.915 & 1 & 1 & 0.068 & 0 & 0.095 & 1333.203\\
0.9 & 0.444 & 0.060 & 0.840 & 1 & 1 & 0.092 & 0 & 0.075 & 1242.407\\
1.0 & 0.493 & 0.050 & 0.690 & 1 & 1 & 0.116 & 0 & 0.210 & 1130.916\\
\bottomrule
\end{tabular}

}

\end{table}%

\begin{table}

\caption{\label{tbl-img-no-intercept-u-any}Share of any imaginary SE in
simulations for HAC+Bsplines with NO intercept in OLS. Uniform Kernel
HAC. Morgan's Locations and Splines.}

\centering{

\begin{tabular}[t]{rrrrrrrrrr}
\toprule
\multicolumn{10}{c}{Any} \\
\cmidrule(l{3pt}r{3pt}){1-10}
\multicolumn{2}{c}{ } & \multicolumn{4}{c}{Unif-HAC+Morgan Splines} \\
\cmidrule(l{3pt}r{3pt}){3-6}
...rho. & corr & 0.05 & 0.1 & 0.15 & 0.2 & NN & Drop & HR & BIC\\
\midrule
0.0 & 0.000 & 0.370 & 1.000 & 1.000 & 1 & -0.103 & 0 & 0.045 & 1752.866\\
0.2 & 0.099 & 0.300 & 0.970 & 0.990 & 1 & -0.048 & 0 & 0.070 & 1676.155\\
0.4 & 0.197 & 0.270 & 0.900 & 0.995 & 1 & 0.014 & 0 & 0.130 & 1583.764\\
0.6 & 0.296 & 0.130 & 0.870 & 0.990 & 1 & 0.057 & 0 & 0.260 & 1475.356\\
0.7 & 0.345 & 0.095 & 0.805 & 0.995 & 1 & 0.095 & 0 & 0.345 & 1407.821\\
\addlinespace
0.8 & 0.394 & 0.045 & 0.805 & 0.995 & 1 & 0.118 & 0 & 0.460 & 1336.252\\
0.9 & 0.444 & 0.040 & 0.740 & 0.985 & 1 & 0.142 & 0 & 0.475 & 1250.878\\
1.0 & 0.493 & 0.035 & 0.610 & 0.990 & 1 & 0.163 & 0 & 0.545 & 1143.030\\
\bottomrule
\end{tabular}

}

\end{table}%

\subsection{Morgan's Locations and
Splines}\label{morgans-locations-and-splines}

\begin{table}

\caption{\label{tbl-HAC-8x8-u-slope}Rejection frequencies testing the
null hypothesis that the slope is statically different from the true
value, zero, the \textbf{uniform} kernel HAC variance estimator for the
standard error. Morgan's 8x8 Splines and Locations. 1000 simulations.
\textbf{500} points. Column \emph{corr } shows the theoretical
correlation at distance \(h=0.1\), thus,
\(corr=\rho*\exp(-\frac{1}{\sqrt{2}})\). HR shows the rejection
frequencies of the Heteroscedastic Robust Variance estimator (Stata).
Negative elements in the diagonal of the variance fixed a la Cameron,
Gelbach and Miller (2011)}

\centering{

\begin{tabular}[t]{rrrrrrrrrr}
\toprule
\multicolumn{10}{c}{Slope} \\
\cmidrule(l{3pt}r{3pt}){1-10}
\multicolumn{2}{c}{ } & \multicolumn{4}{c}{Unif-HAC+Morgans Splines} \\
\cmidrule(l{3pt}r{3pt}){3-6}
...rho. & corr & 0.05 & 0.1 & 0.15 & 0.2 & NN & Drop & HR & BIC\\
\midrule
0.0 & 0.000 & 0.048 & 0.016 & 0.009 & 0.008 & -0.104 & 0 & 0.054 & 1758.853\\
0.2 & 0.099 & 0.043 & 0.013 & 0.010 & 0.005 & -0.062 & 0 & 0.052 & 1681.842\\
0.4 & 0.197 & 0.052 & 0.023 & 0.016 & 0.014 & -0.019 & 0 & 0.054 & 1588.585\\
0.6 & 0.296 & 0.050 & 0.028 & 0.018 & 0.014 & 0.023 & 0 & 0.056 & 1476.531\\
1.0 & 0.493 & 0.124 & 0.083 & 0.048 & 0.046 & 0.114 & 0 & 0.173 & 1132.006\\
\bottomrule
\end{tabular}

}

\end{table}%

\begin{table}

\caption{\label{tbl-HAC-8x8-u-ci-unif-bs-morgan}Confidence Interval
length of using the \textbf{uniform} kernel HAC variance estimator for
the standard error after adding 8x8 B-splines. The number of B-splines
was fixed at 8x8. 1000 simulations. \textbf{500} points. Column
\emph{corr} shows the theoretical correlation at distance \(h=0.1\),
thus, \(corr=\rho*\exp(-\frac{1}{\sqrt{2}})\). HR shows the rejection
frequencies of the Heteroscedastic Robust Variance estimator (Stata's).
Morgan's locations and splines. Negative elements in the diagonal of the
variance fixed a la Cameron, Gelbach and Miller (2011)}

\centering{

\begin{tabular}[t]{rrrrrrrrrr}
\toprule
\multicolumn{10}{c}{Slope} \\
\cmidrule(l{3pt}r{3pt}){1-10}
\multicolumn{2}{c}{ } & \multicolumn{4}{c}{Unif-HAC+Morgans Splines} \\
\cmidrule(l{3pt}r{3pt}){3-6}
...rho. & corr & 0.05 & 0.1 & 0.15 & 0.2 & NN & Drop & HR & BIC\\
\midrule
0.0 & 0.000 & 0.200 & 0.252 & 0.273 & 0.287 & -0.104 & 0 & 0.054 & 1758.853\\
0.2 & 0.099 & 0.197 & 0.244 & 0.265 & 0.279 & -0.062 & 0 & 0.052 & 1681.842\\
0.4 & 0.197 & 0.194 & 0.236 & 0.259 & 0.271 & -0.019 & 0 & 0.054 & 1588.585\\
0.6 & 0.296 & 0.194 & 0.231 & 0.253 & 0.263 & 0.023 & 0 & 0.056 & 1476.531\\
1.0 & 0.493 & 0.233 & 0.260 & 0.297 & 0.307 & 0.114 & 0 & 0.173 & 1132.006\\
\bottomrule
\end{tabular}

}

\end{table}%

\begin{table}

\caption{\label{tbl-img-slope-u}Share of imaginary se in simulations for
HAC+Bsplines. Slope. Uniform Kernel HAC. Morgan's Locations and Splines.
Negative elements in the diagonal of the variance fixed a la Cameron,
Gelbach and Miller (2011)}

\centering{

\begin{tabular}[t]{rrrrrrrrrr}
\toprule
\multicolumn{10}{c}{Slope} \\
\cmidrule(l{3pt}r{3pt}){1-10}
\multicolumn{2}{c}{ } & \multicolumn{4}{c}{Unif-HAC+Morgan Splines} \\
\cmidrule(l{3pt}r{3pt}){3-6}
...rho. & corr & 0.05 & 0.1 & 0.15 & 0.2 & NN & Drop & HR & BIC\\
\midrule
0.0 & 0.000 & 0 & 0 & 0 & 0 & -0.104 & 0 & 0.054 & 1758.853\\
0.2 & 0.099 & 0 & 0 & 0 & 0 & -0.062 & 0 & 0.052 & 1681.842\\
0.4 & 0.197 & 0 & 0 & 0 & 0 & -0.019 & 0 & 0.054 & 1588.585\\
0.6 & 0.296 & 0 & 0 & 0 & 0 & 0.023 & 0 & 0.056 & 1476.531\\
1.0 & 0.493 & 0 & 0 & 0 & 0 & 0.114 & 0 & 0.173 & 1132.006\\
\bottomrule
\end{tabular}

}

\end{table}%

\begin{figure}

\begin{minipage}{\linewidth}

\begin{figure}[H]

\centering{

\includegraphics{Theta-v2_files/figure-pdf/fig-morgans-splines-1.pdf}

}

\caption{\label{fig-morgans-splines-1}Morgan's splines horizontal
(probably)}

\end{figure}%

\end{minipage}%
\newline
\begin{minipage}{\linewidth}

\begin{figure}[H]

\centering{

\includegraphics{Theta-v2_files/figure-pdf/fig-morgans-splines-2.pdf}

}

\caption{\label{fig-morgans-splines-2}Morgan's splines vertical
(probably)}

\end{figure}%

\end{minipage}%

\end{figure}%

\section{8x8 Quadratic Splines
(Matlab)}\label{x8-quadratic-splines-matlab}

\begin{table}

\caption{\label{tbl-HAC-8x8-u-Quad-bs-Morgan}Rejection frequencies
testing the null hypothesis that the slope is statically different from
the true value, zero, using an 8x8 \textbf{quadratic} B-splines and the
\textbf{uniform} kernel HAC variance estimator for the standard error.
The number of B-splines was fixed at 8x8. 1000 simulations. \textbf{500}
points. Column \emph{corr} shows the theoretical correlation at distance
\(h=0.1\), \(corr=\rho*\exp(-\frac{1}{\sqrt{2}})\). HR shows the
rejection frequencies of the Heteroscedastic Robust Variance estimator
(Stata's). Morgan's locations (Poisson Points).}

\centering{

\begin{tabular}[t]{rrrrrrrrrr}
\toprule
\multicolumn{10}{c}{Slope} \\
\cmidrule(l{3pt}r{3pt}){1-10}
\multicolumn{2}{c}{ } & \multicolumn{4}{c}{Unif-HAC+Quad-BS} \\
\cmidrule(l{3pt}r{3pt}){3-6}
...rho. & corr & 0.05 & 0.1 & 0.15 & 0.2 & NN & Drop & HR & BIC\\
\midrule
0.0 & 0.000 & 0.046 & 0.050 & 0.065 & 0.077 & -0.165 & 0 & 0.057 & 1938.790\\
0.2 & 0.099 & 0.033 & 0.034 & 0.051 & 0.069 & -0.126 & 0 & 0.042 & 1852.769\\
0.4 & 0.197 & 0.042 & 0.038 & 0.055 & 0.067 & -0.084 & 0 & 0.056 & 1747.624\\
0.6 & 0.296 & 0.053 & 0.044 & 0.047 & 0.074 & -0.043 & 0 & 0.054 & 1617.099\\
1.0 & 0.493 & 0.080 & 0.051 & 0.051 & 0.062 & 0.042 & 0 & 0.112 & 1161.738\\
\bottomrule
\end{tabular}

}

\end{table}%

\begin{table}

\caption{\label{tbl-HAC-8x8-u-ci-unif-bs}Confidence Interval length of
using the \textbf{uniform} kernel HAC variance estimator for the
standard error after adding 8x8 \textbf{quadratic} B-splines. The number
of B-splines was fixed at 8x8. 1000 simulations. \textbf{500} points.
Column \emph{corr} shows the theoretical correlation at distance
\(h=0.1\), thus, \(corr=\rho*\exp(-\frac{1}{\sqrt{2}})\). HR shows the
rejection frequencies of the Heteroscedastic Robust Variance estimator
(Stata's). Morgan's locations (Poisson Points).}

\centering{

\begin{tabular}[t]{rrrrrrrrrr}
\toprule
\multicolumn{10}{c}{Slope} \\
\cmidrule(l{3pt}r{3pt}){1-10}
\multicolumn{2}{c}{ } & \multicolumn{4}{c}{Unif-HAC+Quad-BS} \\
\cmidrule(l{3pt}r{3pt}){3-6}
...rho. & corr & 0.05 & 0.1 & 0.15 & 0.2 & NN & Drop & HR & BIC\\
\midrule
0.0 & 0.000 & 0.207 & 0.210 & 0.205 & 0.199 & -0.165 & 0 & 0.057 & 1938.790\\
0.2 & 0.099 & 0.205 & 0.210 & 0.207 & 0.200 & -0.126 & 0 & 0.042 & 1852.769\\
0.4 & 0.197 & 0.203 & 0.209 & 0.206 & 0.203 & -0.084 & 0 & 0.056 & 1747.624\\
0.6 & 0.296 & 0.202 & 0.210 & 0.208 & 0.203 & -0.043 & 0 & 0.054 & 1617.099\\
1.0 & 0.493 & 0.221 & 0.248 & 0.253 & 0.252 & 0.042 & 0 & 0.112 & 1161.738\\
\bottomrule
\end{tabular}

}

\end{table}%

\subsection{R vs Matlab}\label{r-vs-matlab}

\begin{table}

\caption{\label{tbl-R-results-pc}R's results with Morgan's B-Splines.
Using Morgan's data. HR and Conley for different cutoffs.}

\centering{

\begingroup
\centering
\begin{tabular}{lccccc}
   \tabularnewline \midrule \midrule
   Dependent Variable: & \multicolumn{5}{c}{sim\_y}\\
                  & HR       & Conley0.05    & Conley0.10   & Conley0.15 & Conley0.20 \\   
   Model:         & (1)      & (2)           & (3)          & (4)        & (5)\\  
   \midrule
   \emph{Variables}\\
   Constant       & 0.0720   & 0.0720$^{**}$ & 0.0720$^{*}$ & 0.0720     & 0.0720\\   
                  & (0.0447) & (0.0348)      & (0.0415)     & (0.0439)   & (0.0454)\\   
   sim\_x         & -0.0584  & -0.0584       & -0.0584      & -0.0584    & -0.0584\\   
                  & (0.0478) & (0.0458)      & (0.0478)     & (0.0629)   & (0.0813)\\   
   \midrule
   \emph{Fit statistics}\\
   Observations   & 500      & 500           & 500          & 500        & 500\\  
   R$^2$          & 0.21409  & 0.21409       & 0.21409      & 0.21409    & 0.21409\\  
   Adjusted R$^2$ & 0.09846  & 0.09846       & 0.09846      & 0.09846    & 0.09846\\  
   \midrule \midrule
   \multicolumn{6}{l}{\emph{Signif. Codes: ***: 0.01, **: 0.05, *: 0.1}}\\
\end{tabular}

\par \raggedright

Note: Conley uses lat-lon locations, and cutoff is given in km. Here
0.05 cutoff is 5.5 km; 0.10, 11 km; 0.15, 16.5 km; and 0.20, 22km.

\par\endgroup

}

\end{table}%

\begin{table}

\caption{\label{tbl-matlab-rslts-splines-f}Matlab Results with Morgan's
B-Splines. Negative Variance elements in the diagonal Fixed a la
Cameron, Gelbach, and Miller (2011)}

\centering{

\begin{tabular}[t]{rrrrrr}
\toprule
Coeffs & HR & Conley0.05 & Conley0.10 & Conley0.15 & Conley0.20\\
\midrule
0.0720 & 0.0447 & 0.0353 & 0.0417 & 0.0437 & 0.0440\\
-0.0584 & 0.0478 & 0.0456 & 0.0474 & 0.0657 & 0.0786\\
\bottomrule
\end{tabular}

}

\end{table}%

\begin{table}

\caption{\label{tbl-matlab-rslts-splines-o}Matlab Results with Morgan's
B-Splines. Original.}

\centering{

\begin{tabular}[t]{rrrlll}
\toprule
Coeffs & HR & Conley0.05 & Conley0.10 & Conley0.15 & Conley0.20\\
\midrule
0.0720 & 0.0447 & 0.0353 & 0.0000+0.018i & 0.0000+0.0241i & 0.0000+0.0052i\\
-0.0584 & 0.0478 & 0.0456 & 0.0397+0.000i & 0.0408+0.0000i & 0.0549+0.0000i\\
\bottomrule
\end{tabular}

}

\end{table}%

\begin{table}

\caption{\label{tbl-R-results-bam-gam-pc}R's results. Using Morgan's
data with Morgan's Splines. HR and Conley for different cutoffs. Does
PC, Bam, and Gam matter?}

\centering{

\begingroup
\centering
\begin{tabular}{lccc}
   \tabularnewline \midrule \midrule
   Dependent Variable: & \multicolumn{3}{c}{sim\_y}\\
                  & PC            & BAM           & GAM \\   
   Model:         & (1)           & (2)           & (3)\\  
   \midrule
   \emph{Variables}\\
   Constant       & 0.0720$^{**}$ & 0.0720$^{**}$ & 0.0720$^{**}$\\   
                  & (0.0348)      & (0.0348)      & (0.0348)\\   
   sim\_x         & -0.0584       & -0.0584       & -0.0584\\   
                  & (0.0458)      & (0.0458)      & (0.0458)\\   
   \midrule
   \emph{Fit statistics}\\
   Observations   & 500           & 500           & 500\\  
   R$^2$          & 0.21409       & 0.21409       & 0.21409\\  
   Adjusted R$^2$ & 0.09846       & 0.09846       & 0.09846\\  
   \midrule \midrule
   \multicolumn{4}{l}{\emph{Conley (5.5km) standard-errors in parentheses}}\\
   \multicolumn{4}{l}{\emph{Signif. Codes: ***: 0.01, **: 0.05, *: 0.1}}\\
\end{tabular}

\par \raggedright

Note: Using all 63 basis of either PC, BAM or GAM splines.

\par\endgroup

}

\end{table}%

\section{No Splines}\label{no-splines}

\begin{table}

\caption{\label{tbl-R-results}R's results. Using Morgan's data. No
Splines. HR and Conley for different cutoffs.}

\centering{

\begingroup
\centering
\begin{tabular}{lcccc}
   \tabularnewline \midrule \midrule
   Dependent Variable: & \multicolumn{4}{c}{sim\_y}\\
                  & HR       & Conley0.05 & Conley0.10 & Conley0.15 \\   
   Model:         & (1)      & (2)        & (3)        & (4)\\  
   \midrule
   \emph{Variables}\\
   Constant       & 0.0697   & 0.0697     & 0.0697     & 0.0697\\   
                  & (0.0473) & (0.0559)   & (0.0638)   & (0.0711)\\   
   sim\_x         & -0.0375  & -0.0375    & -0.0375    & -0.0375\\   
                  & (0.0486) & (0.0467)   & (0.0416)   & (0.0369)\\   
   \midrule
   \emph{Fit statistics}\\
   Observations   & 500      & 500        & 500        & 500\\  
   R$^2$          & 0.00112  & 0.00112    & 0.00112    & 0.00112\\  
   Adjusted R$^2$ & -0.00089 & -0.00089   & -0.00089   & -0.00089\\  
   \midrule \midrule
   \multicolumn{5}{l}{\emph{Signif. Codes: ***: 0.01, **: 0.05, *: 0.1}}\\
\end{tabular}

\par \raggedright

Note: Conley uses lat-lon locations, and cutoff is given in km. Here
0.05 cutoff is 5.5 km, 0.10 is 11 km, and 0.15 is 16.5 km

\par\endgroup

}

\end{table}%

\begin{table}

\caption{\label{tbl-matlab-rslts}Matlab Results, No Splines}

\centering{

\begin{tabular}[t]{lrrrrr}
\toprule
  & Coeffs & HR & Conley0.05 & Conley0.10 & Conley0.15\\
\midrule
Constant & 0.0696 & 0.0473 & 0.0562 & 0.0641 & 0.0697\\
sim\_x & -0.0375 & 0.0486 & 0.0470 & 0.0417 & 0.0378\\
\bottomrule
\end{tabular}

}

\end{table}%

\begin{table}

\caption{\label{tbl-R-results-mod}R's results. Using Morgan's data. HR
and Conley for different cutoffs. Changing Cutoff.}

\centering{

\begingroup
\centering
\begin{tabular}{lcccc}
   \tabularnewline \midrule \midrule
   Dependent Variable: & \multicolumn{4}{c}{sim\_y}\\
                  & HR       & Conley0.05 & Conley0.10 & Conley0.15 \\   
   Model:         & (1)      & (2)        & (3)        & (4)\\  
   \midrule
   \emph{Variables}\\
   Constant       & 0.0697   & 0.0697     & 0.0697     & 0.0697\\   
                  & (0.0473) & (0.0561)   & (0.0639)   & (0.0697)\\   
   sim\_x         & -0.0375  & -0.0375    & -0.0375    & -0.0375\\   
                  & (0.0486) & (0.0469)   & (0.0415)   & (0.0378)\\   
   \midrule
   \emph{Fit statistics}\\
   Observations   & 500      & 500        & 500        & 500\\  
   R$^2$          & 0.00112  & 0.00112    & 0.00112    & 0.00112\\  
   Adjusted R$^2$ & -0.00089 & -0.00089   & -0.00089   & -0.00089\\  
   \midrule \midrule
   \multicolumn{5}{l}{\emph{Signif. Codes: ***: 0.01, **: 0.05, *: 0.1}}\\
\end{tabular}

\par \raggedright

Note: Conley uses lat-lon locations, and cutoff is given in km. Here
0.05 cutoff is 5.554 km, 0.10 is 11.108 km, and 0.15 is 16.662 km

\par\endgroup

}

\end{table}%

\begin{table}

\caption{\label{tbl-HAC-8x8-u-slope-no-splines}Rejection frequencies
testing the null hypothesis that the slope is statically different from
the true value, zero, the \textbf{uniform} kernel HAC variance estimator
for the standard error. No Splines. 1000 simulations. \textbf{500}
points. Column \emph{corr} shows the theoretical correlation at distance
\(h=0.1\), thus, \(corr=\rho*\exp(-\frac{1}{\sqrt{2}})\). HR shows the
rejection frequencies of the Heteroscedastic Robust Variance estimator
(Stata's). Morgan's locations (Poisson Points).}

\centering{

\begin{tabular}[t]{rrrrrrrrrr}
\toprule
\multicolumn{10}{c}{Slope} \\
\cmidrule(l{3pt}r{3pt}){1-10}
\multicolumn{2}{c}{ } & \multicolumn{4}{c}{Unif-HAC} \\
\cmidrule(l{3pt}r{3pt}){3-6}
...rho. & corr & 0.05 & 0.1 & 0.15 & 0.2 & NN & Drop & HR & BIC\\
\midrule
0.0 & 0.000 & 0.051 & 0.061 & 0.088 & 0.108 & 0.003 &  & 0.046 & 1429.291\\
0.2 & 0.099 & 0.110 & 0.096 & 0.106 & 0.118 & 0.170 &  & 0.121 & 1420.015\\
0.4 & 0.197 & 0.214 & 0.166 & 0.144 & 0.149 & 0.327 &  & 0.273 & 1406.699\\
0.6 & 0.296 & 0.277 & 0.194 & 0.141 & 0.145 & 0.496 &  & 0.399 & 1397.871\\
1.0 & 0.493 & 0.327 & 0.191 & 0.147 & 0.136 & 0.820 &  & 0.605 & 1372.726\\
\bottomrule
\end{tabular}

}

\end{table}%

\begin{table}

\caption{\label{tbl-HAC-8x8-u-ci-unif}Confidence Interval length of
using the \textbf{uniform} kernel HAC variance estimator for the
standard error. No B-Splines. Column \emph{corr} shows the theoretical
correlation at distance \(h=0.1\), thus,
\(corr=\rho*\exp(-\frac{1}{\sqrt{2}})\). HR shows the rejection
frequencies of the Heteroscedastic Robust Variance estimator (Stata's).
Morgan's locations.}

\centering{

\begin{tabular}[t]{rrrrrrrrrr}
\toprule
\multicolumn{10}{c}{Slope} \\
\cmidrule(l{3pt}r{3pt}){1-10}
\multicolumn{2}{c}{ } & \multicolumn{4}{c}{Unif-HAC} \\
\cmidrule(l{3pt}r{3pt}){3-6}
...rho. & corr & 0.05 & 0.1 & 0.15 & 0.2 & NN & Drop & HR & BIC\\
\midrule
0.0 & 0.000 & 0.174 & 0.171 & 0.166 & 0.159 & 0.003 &  & 0.046 & 1429.291\\
0.2 & 0.099 & 0.181 & 0.188 & 0.190 & 0.187 & 0.170 &  & 0.121 & 1420.015\\
0.4 & 0.197 & 0.200 & 0.231 & 0.248 & 0.252 & 0.327 &  & 0.273 & 1406.699\\
0.6 & 0.296 & 0.230 & 0.288 & 0.317 & 0.327 & 0.496 &  & 0.399 & 1397.871\\
1.0 & 0.493 & 0.311 & 0.431 & 0.489 & 0.507 & 0.820 &  & 0.605 & 1372.726\\
\bottomrule
\end{tabular}

}

\end{table}%



\end{document}
